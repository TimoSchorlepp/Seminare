\documentclass[twoside]{article}
\setlength{\oddsidemargin}{0. in}
\setlength{\evensidemargin}{-0 in}
\setlength{\topmargin}{-0. in}
\setlength{\textwidth}{7 in}
\setlength{\textheight}{8.4 in}
\setlength{\headsep}{0.75 in}
\setlength{\parindent}{0 in}
\setlength{\parskip}{0.05 in}

\usepackage{amsmath,amsfonts,graphicx,amsthm}
\usepackage[utf8]{inputenc}
\usepackage[ngerman]{babel}
\usepackage{enumerate}
\usepackage{xcolor}
\usepackage{todonotes}
\usepackage[left=2cm,right=2cm,top=3cm,bottom=2cm,]{geometry}

\newcounter{lecnum}
\renewcommand{\thepage}{\thelecnum-\arabic{page}}
\renewcommand{\thesection}{\thelecnum.\arabic{section}}
\renewcommand{\theequation}{\thelecnum.\arabic{equation}}
\renewcommand{\thefigure}{\thelecnum.\arabic{figure}}
\renewcommand{\thetable}{\thelecnum.\arabic{table}}

\setcounter{lecnum}{11}

\newcommand{\head}{
   \pagestyle{myheadings}
   \thispagestyle{plain}
   \newpage
   \setcounter{page}{1}
   \noindent
   \begin{center}
   \framebox{
      \vbox{\vspace{2mm}
    \hbox to 6.28in { {\bf Seminar zur Stochastik SoSe 2019
	\hfill Ruhr-Universität Bochum,  24.06.2019} }
       \vspace{4mm}
       \hbox to 6.28in { {\Large \hfill 11. Vage Konvergenz von Punktmaßen \hfill} }
       \vspace{2mm}
       \hbox to 6.28in { {\it Notizen \hfill von Timo Schorlepp} }
      \vspace{2mm}}
   }
   \end{center}
}

\renewcommand{\cite}[1]{[#1]}
\def\beginrefs{\begin{list}%
        {[\arabic{equation}]}{\usecounter{equation}
         \setlength{\leftmargin}{2.0truecm}\setlength{\labelsep}{0.4truecm}%
         \setlength{\labelwidth}{1.6truecm}}}
\def\endrefs{\end{list}}
\def\bibentry#1{\item[\hbox{[#1]}]}

\newtheorem{theorem}{Satz}[lecnum]
\newtheorem{lemma}[theorem]{Lemma}
\newtheorem{proposition}[theorem]{Proposition}
\newtheorem{claim}[theorem]{Behauptung}
\newtheorem{corollary}[theorem]{Korollar}
\theoremstyle{definition}
\newtheorem{remark}[theorem]{Bemerkung}
\newtheorem{definition}[theorem]{Definition}
%\newenvironment{proof}{{\bf Beweis:}}{\hfill\rule{2mm}{2mm}}
%%%%%%%%%%%%%%%%%%%%%%%%%%%%%%%%%%%%%%%%%%%%%%%%%%%%%
\begin{document}

\head
%%%%%%%%%%%%%%%%%%%%%%%%%%%%%%%%%%%%%%%%%%%%%%%%%%%%%
\textbf{Einleitung:} In diesem Vortrag wollen wir in Vorbereitung auf die Beschäftigung mit schwacher Konvergenz von Punktprozessen den Raum aller Punktmaße auf $E$, bezeichnet mit $M_p(E)$, derart mit einem Konvergenzbegriff ausstatten und dementsprechend topologisieren (''vage Konvergenz/Topologie''), dass $M_p(E)$ zu einem vollständigen, separablen metrischen Raum wird. Dies ist genau das benötigte Setting für die Diskussion von schwacher Konvergenz von Maßen auf $M_p(E)$ (vgl.\ Vorlesung Wahrscheinlichkeitstheorie 1, bspw.\ Skript [3]). Wir werden dabei alle Resultate für den Raum \textit{aller} Radon-Maße $M_+(E)$ formulieren und entsprechende Aussagen für die, wie sich zeigen wird, abgeschlossene Teilmenge $M_p(E)$ folgern. Nach einer Einführung der vagen Topologie werden wir zunächst ein Portmanteau-Theorem zur Charakterisierung von vager Konvergenz, analog zu ähnlichen Resultaten für die schwache Konvergenz, beweisen, und am Ende eine abzählbare Basis der vagen Topologie und eine zugehörige Metrik auf $M_+(E)$ zu finden. Die Darstellung folgt größtenteils [1].
%%%%%%%%%%%%%%%%%%%%%%%%%%%%%%%%%%%%%%%%%%%%%%%%%%%%%
\begin{definition}
(Setting)\\
Sei $E$ ein lokal kompakter, zweitabzählbarer $T_2$-Raum mit Borel-$\sigma$-Algebra $\mathcal{E}$. Bezeichne mit $M_+(E)$ die Menge aller Radon-Maße auf E und wähle
\begin{align}
\mathcal{M}_+(E):=\sigma\left(\left\{\left\{\mu \in M_+(E)|\mu(f) \in B \right\}|f \in C_K^+(E),B \in \mathcal{B}\left[0,\infty \right) \right\}\right)
\end{align}
als $\sigma$-Algebra auf $M_+(E)$, wobei $C_K^+(E):=\left\{ f:E \to [0,\infty) | f \text{ stetig, } \text{supp}(f) \text{ kompakt}\right\}$ die Menge aller stetigen, nichtnegativen Funktionen mit kompaktem Träger auf $E$ ist und $ev_f(\mu) = \mu(f) := \int_E f \; d\mu$.
\end{definition}
%%%%%%%%%%%%%%%%%%%%%%%%%%%%%%%%%%%%%%%%%%%%%%%%%%%%%
\begin{remark}.
\begin{itemize}
\item Die Voraussetzungen an $E$ sind unter anderem so gewählt, dass $E$ gemäß Metrisierbarkeitssatz von Urysohn zu einem separablen metrischen Raum $(E,\rho)$ werden kann.
\item Außerdem garantieren die Voraussetzungen die Existenz einer abzählbaren Basis der Topologie auf $E$ aus relativ kompakten Mengen.
\item Die Voraussetzungen sind topologisch gewählt, um die Unabhängigkeit der Ergebnisse von der konkreten Metrik zu verdeutlichen.
\item Es wird oBdA in der Definition $C_K^+(E)$ statt $C_K(E)$ betrachtet, da sich $f \in C_K(E)$ als $f = f^+ - f^-$ schreiben lässt mit $f^+,f^- \in C_K^+(E)$.
\item Die $\sigma$-Algebra soll also alle Abbildung $ev_f:M_+(E) \to [0,\infty), \mu \mapsto \mu(f)$ messbar machen.
\end{itemize}
\end{remark}
%%%%%%%%%%%%%%%%%%%%%%%%%%%%%%%%%%%%%%%%%%%%%%%%%%%%%
\begin{lemma} (Existenz von Höckerfunktionenfolgen, vgl.\ Lemma 10.2 in [3])\\
(a) Sei $K \subseteq E$ kompakt. Dann existieren kompakte Mengen $K_n \downarrow K$ und eine monoton fallende Folge $(f_n)_{n \in \mathbb{N}}$ mit $f_n \in C_K^+(E) \, \forall \, n \in \mathbb{N}$ mit $1_K \leq f_n \leq 1_{K_n} \downarrow 1_K$.\\
(b) Sei $G \subseteq E$ offen und relativ kompakt. Dann existieren offene und relativ kompakte Mengen $G_n \uparrow G$ und eine monoton wachsende Folge $(g_n)_{n \in \mathbb{N}}$ mit $g_n \in C_K^+(E) \, \forall \, n \in \mathbb{N}$ mit $1_G \geq g_n \geq 1_{G_n} \uparrow 1_G$.
\end{lemma}

\begin{proof} \textcolor{red}{(Weglassen im Vortrag!)}\\
(a) Aus WT1 ist bekannt, dass die Abbildung $x \mapsto \rho(x,K) := \inf_{y \in K} \rho(x,y) =  \min_{y \in K} \rho(x,y)$ stetig ist, dass $\rho(x,K)=0 \Leftrightarrow x \in \overline{K} = K$ und dass für alle $\epsilon > 0$ die Menge $K^\epsilon := \{x \in E | \rho(x,K) \leq \epsilon\}$ abgeschlossen ist. Mit der Hilfsfunktion $\phi:\mathbb{R} \to \mathbb{R}$,
\begin{align}
\phi(t) := \begin{cases}
1 &\; , \text{falls } t < 0\\
1-t &\; , \text{falls } 0 \leq t < 1\\
0 &\; , \text{falls } t > 1
\end{cases}
\end{align}
ist dann $\phi\left(\frac{1}{\epsilon} \rho(x,K) \right)$ die gesuchte Höckerfunktion zu $K^\epsilon$. Es bleibt also zu zeigen, dass sich eine Folge von kompakten $K^{\epsilon_n}$ finden lässt mit $\epsilon_n \to 0$. Sei dazu $(B_k)$ eine offene, relativ kompakte Ausschöpfung von $E$ (nutze dazu die abzählbare Basis von $E$ und bilde für $B_k$ die Vereinigung der ersten $k$ Basismengen; endliche Vereinigungen kompakter Mengen sind immer noch kompakt, und der Abschluss einer endlichen Vereinigung ist die Vereinigung der Abschlüsse, also sind endliche Vereinigungen der erhaltenen Basismengen relativ kompakt). Da $K$ kompakt ist und von $(B_k)$ überdeckt wird, existiert eine endliche Teilüberdeckung, also auch ein maximaler Index $k_0$, sodass $K \subseteq B_{k_0}$. Es gilt $\rho(K,B_{k_0}^c) = \epsilon_0 > 0$, da ansonsten (da beide Mengen abgeschlossen sind) ein $x \in K \cap B_{k_0}^c$ existieren würde, im Widerspruch zu $K \subseteq B_{k_0}$. Wähle dann die Nullfolge $\epsilon_n = \frac{\epsilon_0}{n}$. Die so erhaltenen $K^{\epsilon_n}$ sind kompakt als abgeschlossene Teilmengen der kompakten Menge $\overline{B}_{k_0}$.\\

(b) Als offene Menge in einem metrischen Raum ist $G$ eine $F_\sigma$-Menge, das heißt die abzählbare Vereinigung von abgeschlossenen Mengen $K_n$ (wähle konkret $K_n=\left\{x \in G | \rho(x,G^c) \geq 2^{-n} \right\}$). Da $G$ relativ kompakt ist, sind alle $K_n$ kompakt, und außerdem ist $(K_n)$ per Konstruktion aufsteigend gegen $G$. Definiert man $G_n = \stackrel{\circ}{K}_n$, so erhält man also eine gegen $G$ aufsteigende Folge offener und relativ kompakter Mengen mit $\rho(G_n,G^c) > 0 \; \forall n \in \mathbb{N}$ per Konstruktion. Setze dann
\begin{align}
g_n(x)= \phi \left( \frac{\rho(x,G_n)}{\rho(G^c,G_n)} \right).
\end{align}
Dann gilt $g_n(x) = 1$ für $x \in G_n$, $g_n(x) = 1$ für $x \in G$ und $n$ groß genug, sowie $g_n(x) = 0$ für $x \in G^c$. Außerdem ist die Folge für festes $x$ monoton wachsend, da
\begin{align}
\rho(x,G_n) = \inf_{y \in G_n} \rho(x,y) \geq  \inf_{y \in G_{n+1}} \rho(x,y) = \rho(x,G_{n+1}),
\end{align} 
denn $G_n \subseteq G_{n+1}$, also ist das Argument von $\phi$ monoton fallend.
\end{proof}
%%%%%%%%%%%%%%%%%%%%%%%%%%%%%%%%%%%%%%%%%%%%%%%%%%%%%
\begin{proposition} (vgl.\ Satz 10.3 in [3])\\
$C_K^+(E)$ ist eine trennende Familie für $M_+(E)$, d.\ h.\ für $\mu, \nu \in M_+(E)$ gilt: 
\begin{align}
\forall \, f \in C_K^+(E) : \, \mu(f) = \nu(f) \Longrightarrow \mu = \nu.
\end{align}
\end{proposition}

\begin{proof} \textcolor{red}{(Weglassen im Vortrag!)}\\
Wähle eine abzählbare Basis $\mathcal{A}$ aus relativ kompakten Mengen für die Topologie auf $E$, dann gilt $\mathcal{E} = \sigma(\mathcal{A})$. Dann genügt es also, dass $\mu(A) = \nu(A) \, \forall \, A \in \mathcal{A}$, damit $\mu(A) = \nu(A)$ auf $\mathcal{E}$ (Eindeutigkeitssatz für Maße, wegen der Bedingungen an $E$ sind alle Radon-Maße auf $E$ $\sigma$-endlich, und $\mathcal{A}$ ist schnittstabil, da Teilmengen relativ kompakter Mengen wieder relativ kompakt sind). Sei also $A \in \mathcal{A}$. Nach dem vorigen Lemma exisitiert eine monoton wachsende Folge $g_n \uparrow 1_A$, und somit gilt mit dem Satz über die majorisierte Konvergenz mit Majorante $1_A$ ($\mu(A),\nu(A) < \infty$, da Radon!):
\begin{align}
\mu(A) = \mu(1_A) = \mu \left(\lim_{n \to \infty} g_n \right) =  \lim_{n \to \infty} \mu(g_n) = \lim_{n \to \infty} \nu(g_n)  = \nu \left(\lim_{n \to \infty} g_n \right) = \nu(1_A) = \nu(A)
\end{align}
\end{proof}
%%%%%%%%%%%%%%%%%%%%%%%%%%%%%%%%%%%%%%%%%%%%%%%%%%%%%
\begin{definition} 
(Vage Konvergenz, gerechtfertigt durch die vorherige Proposition)\\
Sei $(\mu_n)_{n \in \mathbb{N}}$ eine Folge in $M_+(E)$ und $\mu \in M_+(E)$. Wir sagen, dass die Folge $(\mu_n)$ vage gegen $\mu$ konvergiert (geschrieben $\mu_n \xrightarrow{v} \mu$), wenn
$
\forall f \in C_K^+(E): \; \mu_n(f) \xrightarrow{} \mu(f)
$.\\
Wir topologisieren $M_+(E)$ unter diesem Konvergenzbegriff, das heißt eine Subbasis ist gegeben durch Mengen der Form $\left\{\mu \in M_+(E) | s < \mu(f) < t \right\}$ für $f \in C_K^+(E)$ und $s,t \in \mathbb{R}, s < t$, und die vage Topologie ist damit die durch alle Auswertungsabbildungen $ev_f:M_+(E) \to [0,\infty), \mu \mapsto \mu(f)$ induzierte Topologie.
\end{definition}
%%%%%%%%%%%%%%%%%%%%%%%%%%%%%%%%%%%%%%%%%%%%%%%%%%%%%
\begin{remark}.
\begin{itemize}
\item $\mu_n \xrightarrow{w} \mu \Longrightarrow \mu_n \xrightarrow{v} \mu$, aber die Umkehrung ist i.\ A.\ falsch, betrachte dazu $E=\mathbb{R}$, $\mu_n = \delta_n$ und $f(x)=\sin(x)$ als stetige und beschränkte Testfunktion. Vage Konvergenz stellt eine Art lokale Version der schwachen Konvergenz dar. 
\item Der Raum $M_+(E)$ mit der vagen Topologie ist homöomorph zu seinem Bild unter der Abbildung $\Lambda: M_+(E) \to [0,\infty)^{C_K^+(E)}, \mu \mapsto (\mu(f))_{f \in C_K^+(E)}$, wobei der Bildraum mit der Produkttopologie ausgestattet wird, also der Topologie der punktweisen Konvergenz. Dies folgt direkt aus der Konstruktion der Topologien und der universellen Eigenschaft der Produkt- und induzierten Topologie.
\item Erstaunlicherweise ist die vage Topologie dennoch metrisierbar, wie wir am Ende des Vortrags sehen werden. Nicht beweisen werden wir, dass $M_+(E)$ ein sequentieller Raum ist. Dieses Detail wird im Beweis der Metrisierbarkeit entscheidend benötigt, findet sich aber in keiner der beiden klassischen Referenzen [1], [2] und wurde erst 2018 in [4] in der Literatur angesprochen. Dass Topologien sequentiell sind, garantiert uns, dass es genügt, die Gleichheit des Begriffs der Folgenkonvergenz zu zeigen, um die Gleichheit der Topologien nachzuweisen. 
\end{itemize}
\end{remark}
%%%%%%%%%%%%%%%%%%%%%%%%%%%%%%%%%%%%%%%%%%%%%%%%%%%%%
\begin{proposition}
Es gilt $\mathcal{M}_+(E) = \mathcal{B}(M_+(E))$.
\end{proposition}

\begin{proof}
"$\subseteq$": Alle Abbildungen $\mu \mapsto \mu(f)$ für $f \in C_K^+(E)$ sind $\mathcal{B}(M_+(E))$-$\mathcal{B}([0,\infty))$-messbar (da stetig). Per Konstruktion ist $\mathcal{M}_+(E)$ aber die \textit{kleinste} $\sigma$-Algebra, bezüglich der diese Abbildungen alle messbar sind.\\

"$\supseteq$": Zeige, dass alle offenen Mengen der vagen Topologie $\mathcal{M}_+(E)$-messbar sind. Leider ergeben sich nicht alle offenen Mengen als Urbilder von offenen Mengen in $[0, \infty)$ unter den Auswertungsabbildungen, diese bilden nur eine $\mathcal{M}_E$-messbare Subbasis für die vage Topologie. Zu zeigen bleibt also, dass \textit{alle} offenen Mengen, die sich zunächst als beliebige Vereinigungen von endlichen Durchschnitten der Basismengen ergeben, $\mathcal{M}_+(E)$-messbar sind. Dies folgt nur dann mithilfe der $\sigma$-Additivität von $\mathcal{M}_+(E)$, falls die vage Topologie eine $\mathcal{M}_E$-messbare, abzählbare Basis besitzt; dies wird im Beweis des letzten Satzes gezeigt.
\end{proof}
%%%%%%%%%%%%%%%%%%%%%%%%%%%%%%%%%%%%%%%%%%%%%%%%%%%%%
\begin{lemma}(Urbild von Rändern) \textcolor{red}{(Weglassen im Vortrag!)}\\
Seien $E_1$ und $E_2$ topologische Räume sowie $T:E_1 \to E_2$ stetig. Dann gilt für $A \subseteq E_2$: $\partial \left(T^{-1} A \right) \subseteq T^{-1} \left(\partial A \right)$.
\end{lemma}

\begin{remark} \textcolor{red}{(Weglassen im Vortrag!)}\\
Die andere Inklusion gilt i.\ A.\ nicht, betrachte dazu $T:\mathbb{R} \to \mathbb{R}, T(x) \equiv 0$ sowie $A = \{0\}$.
\end{remark}

\begin{proof} \textcolor{red}{(Weglassen im Vortrag!)}\\
Es gilt immer $T^{-1} (\partial A) = T^{-1} \left( \overline{A} \setminus \stackrel{\circ}{A} \right) = T^{-1}(\overline{A}) \setminus T^{-1}\left(\stackrel{\circ}{A} \right)$. Da $\overline{A} \supseteq A$, gilt $T^{-1}(\overline{A}) \supseteq T^{-1}(A)$, aber da $T$ stetig ist, ist $T^{-1}(\overline{A})$ abgeschlossen und somit $T^{-1}(\overline{A}) \supseteq \overline{T^{-1}(A)}$. Analog folgt $T^{-1}(\stackrel{\circ}{A}) \subseteq \left(T^{-1}(A)\right)^\circ$ und somit
$ T^{-1} (\partial A) \supseteq \overline{T^{-1}(A)} \setminus \left(T^{-1}(A)\right)^\circ = \partial \left(T^{-1}(A) \right)$.
\end{proof}
%%%%%%%%%%%%%%%%%%%%%%%%%%%%%%%%%%%%%%%%%%%%%%%%%%%%%
\begin{theorem} (Portmanteau-Theorem für vage Konvergenz, vgl.\ Satz 10.6 in [3] für schwache Konvergenz)\\
Sei $(\mu_n)_{n \in \mathbb{N}}$ eine Folge in $M_+(E)$ und $\mu \in M_+(E)$. Dann sind folgende Aussagen äquivalent:
\begin{enumerate}[(i)]
\item $\mu_n \xrightarrow{v} \mu$
\item $\mu_n(B) \to \mu(B)$ für alle relativ kompakten Mengen $B \in \mathcal{E}$ mit $\mu(\partial B) = 0$.
\item $\limsup_{n \to \infty} \mu_n(K) \leq \mu(K)$ für alle kompakten $K \in \mathcal{E}$ und $\liminf_{n \to \infty} \mu_n(G) \geq \mu(G)$ für alle offenen, relativ kompakten $G \in \mathcal{E}$.
\end{enumerate}
\end{theorem}

\begin{proof}
$(i) \Longrightarrow (iii)$: (in etwa analog zu [3], Beweis von Satz 10.6 $(ii) \Longrightarrow (iii)$)\\
Für $K$ kompakt existiert nach dem entsprechenden Lemma eine Folge kompakter Mengen $K_n \downarrow K$ und $f_n \in C_K^+(E)$ mit $1_K \leq f_n \leq 1_{K_n} \downarrow 1_K$. Dann gilt für festes $m \in \mathbb{N}$
\begin{align}
\limsup_{n \to \infty} \mu_n(K) = \limsup_{n \to \infty} \mu_n(1_K) \leq \limsup_{n \to \infty} \mu_n(f_m) = \mu(f_m) ,
\end{align}
wobei die letzte Gleichheit aus der Voraussetzung $\mu_n \xrightarrow{v} \mu$ folgt. Wegen des Satzes von der dominierten Konvergenz (Majorante $1_{K_{m_0}}$) gilt aber
\begin{align}
\lim_{m \to \infty} \mu(f_m) = \mu\left(\lim_{m \to \infty} f_m \right) = \mu(1_K) = \mu(K),
\end{align}
also folgt 
\begin{align}
\limsup_{n \to \infty} \mu_n(K) \leq \mu(K).
\end{align}
Sei nun analog $G$ offen und relativ kompakt, $G_n \uparrow G$ Folge offener, relativ kompakter Mengen sowie $g_n \in C_K^+(E)$ mit $1_G \geq g_n \geq 1_{G_n} \uparrow 1_G$. Dann gilt für $m \in \mathbb{N}$ und mit Majorante $1_G$
\begin{align}
\liminf_{n \to \infty} \mu_n(G) = \liminf_{n \to \infty} \mu_n(1_G) \geq \liminf_{n \to \infty} \mu_n(g_m) = \mu(g_m) \xrightarrow{m \to \infty} \mu(G).
\end{align}\\

$(iii) \Longrightarrow (ii)$: (analog zu [3], Beweis von Satz 10.6 $(iii) \Longrightarrow (v)$)\\
Sei $B$ relativ kompakt mit $\mu(\partial B) = 0$. Es gilt immer $\stackrel{\circ}{B} \; \subseteq B \subseteq \overline{B}$, sodass aus $\mu(\partial B) = 0$ sowie $\overline{B} = \stackrel{\circ}{B}  \dot{\cup} \; \partial B$ folgt, dass $\mu(\stackrel{\circ}{B}) = \mu(B) = \mu(\overline{B})$. Damit folgt dann aus der Ungleichungskette
\begin{align}
\mu(\stackrel{\circ}{B}) 
\leq \liminf_{n \to \infty} \mu_n(\stackrel{\circ}{B}) 
\leq \liminf_{n \to \infty} \mu_n(B) 
\leq \limsup_{n \to \infty} \mu_n(B)
\leq \limsup_{n \to \infty} \mu_n(\overline{B})
\leq \mu(\overline{B})
\end{align}
die gewünschte Aussage $\lim_{n \to \infty} \mu_n(B) = \mu(B)$. Dabei wurden beide Teile der Voraussetzung $(iii)$ genutzt, einerseits für die erste Ungleichung (da der Abschluss von $\stackrel{\circ}{B}$ gleich dem Abschluss von $B$ ist, ist auch das Innere von $B$ relativ kompakt), andererseits für die letzte Ungleichung.\\

$(ii) \Longrightarrow (i)$: (in etwa analog zu [3], Beweis von Satz 10.6 $(iii) \Longrightarrow (i)$)\\
Sei $f \in C_K^+(E)$ und bezeichne mit $F$ den kompakten Träger von $f$ sowie $\hat{f}= f |_F:F \to (0,\infty)$. Wir wollen zeigen, dass $\mu_n(f) \to \mu(f)$. Definiere dazu zunächst für $n \in \mathbb{N}$
\begin{align}
\Gamma_n := \left\{\gamma > 0 \, \middle| \, \mu\left(\left\{x \in F \, \middle| \, \hat{f}(x) = \gamma \right\} \right) > \frac{1}{n} \right\}
\end{align}
sowie $\Gamma := \bigcup_{n \in \mathbb{N}} \Gamma_n$. Die Mengen $\left\{x \in F \, \middle| \, \hat{f}(x) = \gamma \right\} \subseteq F$ sind disjunkt für verschiedene $\gamma$, und da $\mu$ ein Radon-Maß ist, d.h. $\mu(F) < \infty$, folgt, dass es für jedes $n$ nur endlich viele $\gamma$ geben kann mit $\mu\left(\left\{x \in F \, \middle| \, \hat{f}(x) = \gamma \right\} \right) > \frac{1}{n}$. Somit ist $\Gamma$ abzählbar. Die Menge $\Gamma = \left\{ \gamma > 0 \, \middle| \, \mu \circ \hat{f}^{-1}(\{\gamma\})>0 \right\}$ ist dabei die Menge der Atome des transportierten Maßes $\mu \circ \hat{f}^{-1}$ auf $(0,\infty)$.\\
Da $\hat{f}$ stetig und $F$ kompakt ist, gibt es ein $\beta > 0$ mit $0 \leq \hat{f} \leq \beta$. Da $\Gamma$ abzählbar ist, gibt es zu einem beliebigen $\epsilon > 0$ $\alpha_0 , \cdots , \alpha_k \in \Gamma^c$ mit $0 = \alpha_0 < \alpha_1 < \cdots \alpha_k = \beta$ und $\max_{1 \leq i \leq k} (\alpha_i - \alpha_{i-1}) \leq \epsilon$. Dann gilt für alle $x \in F$
\begin{align}
\sum_{i=1}^k \alpha_{i-1} 1_{(\alpha_{i-1},\alpha_i]}\left(\hat{f}(x) \right) \leq \hat{f}(x) \leq \sum_{i=1}^k \alpha_{i} 1_{(\alpha_{i-1},\alpha_i]}\left(\hat{f}(x) \right) \label{eq:indikatorineq}\\
\Longrightarrow \sum_{i=1}^k \alpha_{i-1} \mu\left( \hat{f} \in (\alpha_{i-1},\alpha_i]\right) \leq \mu(\hat{f}) \leq \sum_{i=1}^k \alpha_{i} \mu\left( \hat{f} \in (\alpha_{i-1},\alpha_i]\right) \label{eq:muineq}
\end{align}
sowie wegen der Wahl der $\alpha_i$
\begin{align}
\sum_{i=1}^k (\alpha_i - \alpha_{i-1})\,\mu\left( \hat{f} \in (\alpha_{i-1},\alpha_i]\right) \leq \epsilon \sum_{i=1}^k \mu\left( \hat{f} \in (\alpha_{i-1},\alpha_i]\right) = \epsilon \mu(F). \label{maximalineq}
\end{align}
Außerdem sind die Mengen $\{x \in F \, | \, \hat{f}(x) \in (\alpha_{i-1},\alpha_i] \}$ relativ kompakt als Teilmengen der kompakten Menge $F$, und $\mu$-randlos, denn es gilt wegen des Rand-Lemmas  und $\alpha_{i-1},\alpha_i \in \Gamma^c$
\begin{align}
\mu \left(\partial \left(\hat{f}^{-1} (\alpha_{i-1},\alpha_i]  \right) \right) \leq \mu \left(\hat{f}^{-1} \left(\partial (\alpha_{i-1},\alpha_i]  \right) \right) = \mu\left(\hat{f}^{-1}\{\alpha_{i-1},\alpha_i \} \right) = 0.
\end{align}
Mit diesen Hilfsresultaten folgt jetzt
\begin{align}
\limsup_{n \to \infty} \mu_n(f) = \limsup_{n \to \infty} \mu_n(\hat{f}) \overset{(\ref{eq:indikatorineq})}{\leq} \limsup_{n \to \infty} \sum_{i=1}^k \alpha_i \mu_n\left( \hat{f} \in (\alpha_{i-1},\alpha_i]\right) \overset{(ii)}{=} \sum_{i=1}^k \alpha_i \, \mu\left( \hat{f} \in (\alpha_{i-1},\alpha_i]\right) \\
= \sum_{i=1}^k (\alpha_i-\alpha_{i-1}+\alpha_{i-1}) \, \mu\left( \hat{f} \in (\alpha_{i-1},\alpha_i]\right) \underset{(\ref{maximalineq})}{\overset{(\ref{eq:muineq})}{\leq}} \epsilon \mu(F) + \mu(\hat{f})
\end{align}
und analog
\begin{align}
\liminf_{n \to \infty} \mu_n(f) = \liminf_{n \to \infty} \mu_n(\hat{f}) \geq \liminf_{n \to \infty} \sum_{i=1}^k \alpha_{i-1} \mu_n\left( \hat{f} \in (\alpha_{i-1},\alpha_i]\right) \overset{(ii)}{=} \sum_{i=1}^k \alpha_{i-1} \, \mu\left( \hat{f} \in (\alpha_{i-1},\alpha_i]\right) \\
= \sum_{i=1}^k (\alpha_{i-1}+\alpha_{i}-\alpha_{i}) \, \mu\left( \hat{f} \in (\alpha_{i-1},\alpha_i]\right) \geq -\epsilon \mu(F) + \mu(\hat{f}).
\end{align}
Da $\epsilon > 0$ beliebig war, folgt somit $\lim_{n \to \infty} \mu_n(f) = \mu(f)$.
\end{proof}
%%%%%%%%%%%%%%%%%%%%%%%%%%%%%%%%%%%%%%%%%%%%%%%%%%%%%
\begin{proposition} (Vage Konvergenz von Punktmaßen)\\
Sei $(m_n)_{n \in \mathbb{N}}$ eine Folge in $M_p(E)$ und $m \in M_p(E)$. Dann sind folgende Aussagen äquivalent:
\begin{enumerate}[(i)]
\item $m_n \xrightarrow{v} m$.
\item Für alle $B \in \mathcal{E}$ relativ kompakt mit $m(\partial B) = 0$ gibt es für $n \geq n_0(B)$ eine Nummerierung der Punkte von $m_n$ und $m$ in $B$ mit
\begin{align}
m_n(\cdot \cap B) = \sum_{i=1}^p \delta_{x_i^{(n)}} \; ; \quad m(\cdot \cap B) = \sum_{i=1}^p \delta_{x_i} \label{eq:pktmass}
\end{align}
und $(x_1^{(n)}, \cdots , x_p^{(n)}) \to (x_1, \cdots , x_p)$ in $E^p$.
\end{enumerate}
\end{proposition}

\begin{proof}
$(i) \Longrightarrow (ii)$: Bezeichne mit $p < \infty$ die Anzahl an Punkten von $m$ in $B$ (bzw.\ in $\overset{\circ}{B}$, da $m(\partial B) = 0$). Schreibe $m(\cdot \cap B) = \sum_{i=1}^s c_i \delta_{y_i}$ wobei $c_i$ die Multiplizität der jeweiligen paarweise verschiedenen Punkte $y_i \in \overset{\circ}{B}$ angebe. Wähle (da $\overset{\circ}{B}$ offen und $T_2$) paarweise disjunkte Umgebungen $ G_i \subset \overset{\circ}{B}$ für die $y_i$. Dann folgt $m(\partial G_i) = 0$, und die $G_i$ sind als Teilmengen von $B$ relativ kompakt. Wegen des Portmanteau-Theorems, Aussage $(ii)$, gilt also $\lim_{n \to \infty} m_n(G_i) = c_i$, also für $n$ groß genug $m_n(G_i) = c_i$ (da Werte in $\mathbb{N}$ mit diskreter Topologie). Außerdem gilt analog ab einem gewissen $n_0$: $m_n(B)=m(B)$. Zählt man anschließend wieder jeden Punkt einzeln, folgt also die Darstellung (\ref{eq:pktmass}), und die Konvergenz in $E^p$ folgt, da sich die Umgebungen $G_i$ der $y_i$ beliebig kleiner wählen lassen (nutze bspw. Metrik auf $E$, dann liegen in offenen Bällen $B_{\epsilon}(y_i) \subset G_i$ für $\epsilon \to 0$ für jedes $\epsilon$ alle bis auf endlich viele Folgenglieder).\\

$(ii) \Longrightarrow (i)$: Für $n \geq n_0(B)$ gilt $m_n(B) = m_n(B \cap B) = p = m(B)$, also folgt die vage Konvergenz mit dem Portmanteau-Theorem.
\end{proof}
%%%%%%%%%%%%%%%%%%%%%%%%%%%%%%%%%%%%%%%%%%%%%%%%%%%%%
\begin{lemma} (Approximation kompakter Mengen durch randlose Mengen)\\
Sei $K \subset E$ kompakt und $\mu \in M_+(E)$. Dann exisitert eine reelle Folge $\epsilon_n \downarrow 0$ mit $K^{\epsilon_n} \downarrow K$ und $\mu(\partial K^{\epsilon_n} ) = 0 \, \forall n$.
\end{lemma}

\begin{proof} \textcolor{red}{(Weglassen im Vortrag!)}\\
Aus dem Beweis vom Approximationslemma folgt, dass für kleine $\epsilon \leq \epsilon_0$ $K^\epsilon$ kompakt ist. Da $\mu$ ein Radonmaß ist, gilt also $\mu(K^{\epsilon_0}) < \infty$. Weiterhin ist $\partial K^\epsilon \subseteq \{x \in E \, | \, \rho(x,K)=\epsilon \}$ (denn $x \in \partial K^\epsilon \Leftrightarrow \forall$ Umgebungen $U$ von $x$ ist $U \cap K^\epsilon \neq \emptyset$ und $U \cap \left(K^\epsilon\right)^c \neq \emptyset$. Es folgt also $\rho(x,K) = \epsilon$, da sonst Umgebungen gefunden werden können, die eine der beiden Bedingungen verletzen. Die Umkehrung gilt i.\ A.\ nicht, siehe diskrete Topologie/Metrik auf $\mathbb{R}$ und $K = \{0\}$ und $\epsilon = 1$). Daraus folgt, dass die Menge an Rändern $\{\partial K^\epsilon \, | \, 0 < \epsilon < \epsilon_0 \}$ aus paarweise disjunkten Mengen besteht. Also ist (analoge Argumentation zu Beweis Portemanteau) für alle $n \in \mathbb{N}$ die Menge $\left\{\epsilon \in (0,\epsilon_0] \, \middle| \, \mu \left(\partial K^\epsilon \right) > \frac{1}{n} \right\}$ endlich, d.h. die Menge $\left\{\epsilon \in (0,\epsilon_0] \, \middle| \, \mu \left( \partial K^\epsilon\right)>0 \right\}$ ist abzählbar und liefert die gesuchte Folge $(\epsilon_n)_{n \in \mathbb{N}}$.
\end{proof}
%%%%%%%%%%%%%%%%%%%%%%%%%%%%%%%%%%%%%%%%%%%%%%%%%%%%%
\begin{proposition} 
Die Menge der Punktmaße $M_p(E)$ ist vage abgeschlossen in  $M_+(E)$.
\end{proposition}

\begin{proof}
Sei $(\mu_n)_{n \in \mathbb{N}} \subset M_p(E)$, $\mu \in M_+(E)$ und $\mu_n \xrightarrow{v} \mu$. Es ist zu zeigen, dass $\mu \in M_p(E)$. Sei dazu $G_n \uparrow E$ relativ kompakte Ausschöpfung von $E$ mit $\mu(\partial G_n) = 0 \, \forall n \in \mathbb{N}$ (Nutze hier das vorige Lemma, um aus einer beliebigen relativ kompakten Ausschöpfung eine $\mu$-randlose zu erhalten!). Definiere
\begin{align}
\mathcal{B}_\mu := \left\{B \in \mathcal{E} \, \middle| \, B \text{ relativ kompakt, } \mu(\partial B) = 0 \right\}. \label{eq:bmu}
\end{align}
Das Mengensystem $\mathcal{B}_\mu$ ist schnittstabil, da $B_1 \cap B_2 \subseteq B_1$ und $\partial(B_1 \cap B_2) \subseteq \partial B_1 \cup \partial B_2$. Für $B \in \mathcal{B}_\mu$ gilt mit dem Portmanteau-Theorem $\mu_n(B) \to \mu(B)$, also folgt, da $(\mu_n) \subset M_p(E)$, $\mu(B) \in \mathbb{N}$.  Definiere weiterhin für $n \in \mathbb{N}$
\begin{align}
\mathcal{G}_n := \left\{B \in \mathcal{E} \, \middle| \, \mu(B \cap G_n) \in \mathbb{N} \right\}.
\end{align}
Das Mengensystem $\mathcal{G}_n$ ist ein Dynkin-System (da $\mu( E \cap G_n) = \mu(G_n) \in \mathbb{N}$, $\mu((B_2 \setminus B_1) \cap G_n) = \mu ((B_2 \cap G_n) \setminus (B_1 \cap G_n)) = \mu ((B_2 \cap G_n) - \mu ((B_1 \cap G_n) \in \mathbb{N}$ und Stetigkeit von unten für monotone Klasse). Außerdem gilt $\mathcal{B}_\mu \subseteq \mathcal{G}_n$ wegen der obigen Bemerkungen zu $\mathcal{B}_\mu$. Also folgt mit dem Satz von Dynkin aus WT1, dass $\sigma\left(\mathcal{B}_\mu \right) \subseteq \mathcal{G}_n$ für alle $n \in \mathbb{N}$. Es bleibt noch zu zeigen, dass $\sigma\left(\mathcal{B}_\mu \right) = \mathcal{E}$, dann sind wir fertig, da dann für alle $A \in \mathcal{E}$ und für alle $n \in \mathbb{N}$ gilt, dass $\mu(A \cap G_n) \in \mathbb{N}$, also mit Stetigkeit von unten $\mu(A) = \lim_{n \to \infty} \mu(A \cap G_n) \in \mathbb{N}$. Es gilt aber $\mathcal{E} \subseteq \sigma\left(\mathcal{B}_\mu \right)$, da $\mathcal{E}$ von den kompakten Teilmengen von $E$ erzeugt wird und es gemäß des vorherigen Lemmas zu jeder kompakten Menge $K$ eine Folge $K_n \downarrow K$ in $\mathcal{B}_\mu$ gibt. Also ist jedes $K = \bigcap_n K_n$ in $\sigma\left(\mathcal{B}_\mu \right)$ enthalten. Andererseits ist per Definition (\ref{eq:bmu}) $\mathcal{B}_\mu \subseteq \mathcal{E}$, also auch $\sigma\left(\mathcal{B}_\mu \right) \subseteq \mathcal{E}$.
\end{proof}
%%%%%%%%%%%%%%%%%%%%%%%%%%%%%%%%%%%%%%%%%%%%%%%%%%%%%
\begin{proposition}(Relative Kompaktheitskriterien)\\
Sei $M \subseteq M_+(E)$ oder $M \subseteq M_p(E)$. Dann sind äquivalent:
\begin{enumerate}[(i)]
\item $M$ ist vage relativ kompakt.
\item $\sup_{\mu \in M} \mu(f) < \infty$  für alle $\forall f \in C_K^+(E)$.
\item $\sup_{\mu \in M} \mu(B) < \infty$ für alle relativ kompakten Mengen $B \in \mathcal{E}$.
\end{enumerate}
\end{proposition}

\begin{proof} .\\
Behauptung 1: Es genügt, nur den Fall $M \subseteq M_+(E)$ zu betrachten, die Aussagen für $M \subseteq M_p(E)$ folgen dann sofort, da folgende Äquivalenz gilt: $M \subseteq M_p(E)$ relativ kompakt in Teilraumtopologie $\Leftrightarrow$ $M \subseteq M_p(E) \subseteq M_+(E)$ relativ kompakt in der vagen Topologie auf $M_+(E)$.\\

Beweis von Behauptung 1: \textcolor{red}{(Details weglassen!)} Zeige, dass der Abschluss von $M$ in $M_p(E)$ gleich dem Abschluss von $M$ in $M_+(E)$ ist. Allgemein gilt zunächst für $A \subseteq X \subseteq Y$, dass $\overline{A}|_X = \overline{A}|_Y \cap X$, wobei $\overline{A}|_X$ den Abschluss von $A$ in der Teilraumtopologie bezeichnet (denn: ''$\subseteq$'': $A \subseteq \overline{A}|_Y \Longrightarrow A \subseteq \overline{A}|_Y \cap X$, aber diese Obermenge ist per Definition abgeschlossen in $X$, also gilt per Definition des Abschlusses als Schnitt aller abgeschlossenen Obermengen $\overline{A}|_X \subseteq \overline{A}|_Y \cap X$. ''$\supseteq$'': $\overline{A}|_X$ ist abgeschlossen in $X$, also $\overline{A}|_X = \tilde{A} \cap X$ mit $\tilde{A}$ abgeschlossen in $Y$. Dann also $A \subseteq \overline{A}|_X \subseteq \tilde{A}$, also per Definition des Abschlusses in $Y$: $\overline{A}|_Y \subseteq \tilde{A}$. Schneiden mit $X$ liefert die gewünschte Inklusion $\overline{A}|_Y \cap X \subseteq \overline{A}|_X$). Da nach dem vorherigen Lemma die Menge der Punktmaße $X = M_p(E)$ abgeschlossen in $Y = M_+(E)$ ist, gilt aber hier $\overline{A}|_Y \subseteq X$, also $\overline{A}|_X = \overline{A}|_Y$. Damit folgt nun die zu zeigenden Äquivalenz, da ganz allgemein für die Teilraumtopologie folgende Äquivalenz gilt: $K \subseteq X \subseteq Y$ kompakt in $X$ $\Leftrightarrow$ $K \subseteq X \subseteq Y$ kompakt in $Y$ (eine Richtung folgt sofort durch Stetigkeit der Inklusionsabbildung $i:X \to Y$, die andere Richtung durch Vorgeben einer offenen Überdeckung $(U_i)_{i \in I}$ von $K$ in $X$, also $U_i = \tilde{U}_i \cap X$ mit $(\tilde{U}_i)_{i \in I}$ offener Überdeckung von $K$ in $Y$, Ausnutzen der Kompaktheit dort und anschließendes Schneiden mit $X$.).\\

Behauptung 2: Für alle $f \in C_K^+(E)$ gilt $\sup_{\mu \in M} \mu(f) = \sup_{\mu \in \overline{M}} \mu(f)$; es genügt also, den einfacheren Ausdruck auf der RHS zu betrachten.\\

Beweis von Behauptung 2: Per Definition des Supremums gibt es eine Folge $(\mu_n) \subset \overline{M}$ mit $\mu_n(f) \to \sup_{\mu \in \overline{M}} \mu(f)$. Per Charakterisierung des Rands über nichtleere Schnitte von beliebigen Umgebungen mit $M$ gibt es dazu dann eine Folge $(\nu_n) \subset M$ mit $|\mu_n(f) - \nu_n(f)| < 2^{-n}$ (irgendeine Nullfolge genügt; falls $\mu_n \in M$, so wähle $\nu_n = \mu_n$, ansonsten nutze besagte Eigenschaft für die offene Umgebung $\{\nu \in M_+(E) \, | \, |\mu_n(f) - \nu(f)| < 2^{-n} \}$ von $\mu_n$). Damit gilt dann 
\begin{align}
\sup_{\mu \in M} \mu(f) \geq \nu_m(f) \xrightarrow{m \to \infty} \sup_{\mu \in \overline{M}} \mu(f) \geq \sup_{\mu \in M} \mu(f).
\end{align}

Damit nun zum eigentlichen Beweis:\\

$(i) \Longrightarrow (ii)$: Sei $\overline{M}$ kompakt und $f \in C_K^+(E)$. Per Konstruktion der vagen Topologie ist die Auswertungsabbildung $ev_f:M_+(E) \to [0,\infty) \, , \, \mu \mapsto \mu(f)$ stetig. Somit ist das Bild von $\overline{M}$ unter $ev_f$, d.\ h.\ $\{\mu(f) \, | \, \mu \in \overline{M} \}$, kompakt in $[0,\infty)$ und folglich beschränkt. Somit gilt $\sup_{\mu \in \overline{M}} \mu(f) < \infty$.\\

$(ii) \Longrightarrow (i)$: Sei $\sup_{\mu \in \overline{M}} \mu(f) < \infty$ für alle $f \in C_K^+(E)$. Dann ist $I_f := [0 , \sup_{\mu \in \overline{M}} \mu(f)]$ für alle $f \in C_K^+(E)$ eine kompakte Teilmenge von $[0,\infty)$ und somit ist nach dem Satz von Tychonoff auch $I:= \prod_{f \in C_K^+(E)} I_f$ kompakt in $[0,\infty)^{C_K^+(E)}$ mit der Produkttopologie. Nutze den anfangs definierten Homöomorphismus $\Lambda|_{\overline{M}}:\overline{M} \to \Lambda(\overline{M}) \subseteq I \, , \, \mu \mapsto (\mu(f))_{f \in C_K^+(E)}$. Dann folgt, dass, da $\overline{M}$ abgeschlossen ist, auch $\Lambda(\overline{M})$ abgeschlossen in $I$ ist, aber da $I$ kompakt ist, ist folglich auch $\Lambda(\overline{M})$ kompakt, also auch $\overline{M}$.\\

$(ii) \Longrightarrow (iii)$: Sei $B \in \mathcal{E}$ relativ kompakt, und sei $f_n \downarrow 1_{\overline{B}}$ eine Höckerfunktionsfolge aus $C_K^+(E)$-Funktionen für $\overline{B}$. Dann gilt
\begin{align}
\sup_{\mu \in M} \mu(B) \leq \sup_{\mu \in M} \mu(\overline{B}) \leq \sup_{\mu \in M} \mu(f_1) < \infty.
\end{align}\\

$(iii) \Longrightarrow (ii)$: Seien $f \in C_K^+(E)$ und $F$ der kompakte Träger von $f$ sowie $0 \leq c < \infty$ das Maximum von $f$. Dann gilt
\begin{align}
\sup_{\mu \in M} \mu(f) \leq c \; \sup_{\mu \in M} \mu(F) < \infty.
\end{align}
\end{proof}
%%%%%%%%%%%%%%%%%%%%%%%%%%%%%%%%%%%%%%%%%%%%%%%%%%%%%
\begin{theorem} (Metrisierbarkeit)\\
$M_+(E)$ und $M_p(E)$, ausgestattet mit der vagen Topologie, sind metrisierbar zu vollständigen, separablen metrischen Räumen.
\end{theorem}

\begin{proof} Die Idee des Beweises lautet wie folgt: Finde zunächst eine abzählbare Basis der vagen Topologie, dies zeigt Separabilität (wähle aus jeder nichtleeren Basismenge einen Punkt). Nutze die Funktionen, die zur Konstruktion dieser Basismengen genutzt wurden, um eine Metrik zu definieren, die die vage Topologie induziert. Hierbei genügt es, wie anfangs bemerkt, zu prüfen, dass beide Topologien den gleichen Konvergenzbegriff für Folgen liefern. Zeige dann abschließend Vollständigkeit dieser Metrik.\\

Behauptung: Es genügt, dieses Verfahren für $M_+(E)$ durchzuführen, die analogen Aussagen für $M_p(E)$ folgen dann sofort.\\

Beweis der Behauptung: Jede Teilmenge eines zweitabzähbaren Raums ist wieder zweitabzählbar (mit der Teilraumtopologie), also auch separabel. Außerdem ist jede abgeschlossene Teilmenge eines vollständigen, metrischen Raumes wieder vollständig (Sei dazu $(X,d)$ vollständig und $A \subseteq X$ abgeschlossen sowie $(x_n)$ Cauchy-Folge in $A$. Dann konvergiert die Folge $(x_n)$ wegen der Vollständigkeit von $X$ gegen ein $x \in X$, aber wegen der Abgeschlossenheit von $A$ gilt $x \in A$).\\

Damit nun zum eigentlichen Beweis: Der entscheidende Schritt ist es, eine abzählbare Menge an Funktionen $\{h_l\, | \, \l \in \mathbb{N}\} \subset C_K^+(E)$ zu finden, sodass $\mu_n \xrightarrow{v} \mu \Longleftrightarrow \mu_n(h_l) \to \mu(h_l) \, \forall \, l \in \mathbb{N}$ gilt. Sei dazu $\mathcal{G} := \{ G_i \, | \, i \in \mathbb{N} \}$ eine abzählbare Basis der Topologie auf $E$ aus relativ kompakten Mengen, wobei $\mathcal{G}$ oBdA bereits unter endlichen Vereinigungen abgeschlossen sei. Wegen des Approximationslemmas gibt es Folgen $(f_{n,i})_{n \in \mathbb{N}}, (g_{n,i})_{n \in \mathbb{N}} \subset C_K^+(E)$ mit
\begin{align}
\lim_{n \to \infty} \uparrow g_{i,n} = 1_{G_i} \, , \, \lim_{n \to \infty} \downarrow f_{i,n} = 1_{\overline{G}_i}.
\end{align}
Nutze das Cantorsche Diagonalverfahren, um die Menge $ \{f_{i,n}, g_{i,n} \, | \, i.n \in \mathbb{N} \}$ als $\{h_l \, | \, l \in \mathbb{N} \}$ zu schreiben. Jedes $\mu \in M_+(E)$ wird durch $\{\mu(h_l) \, | \, l \in \mathbb{N}\}$ eindeutig bestimmt, denn es gilt $\mu(G_i) = \lim_{n \to \infty} \mu(f_{i,n})$ (dominierte Konvergenz), und da $\mathcal{G}$ ein schnittstabiler Erzeuger der Borel-$\sigma$-Algebra $\mathcal{E}$ ist und wie bereits erwähnt Radon-Maße auf $E$ $\sigma$-endlich sind, gilt der Eindeutigkeitssatz für Maße. Es bleibt zu zeigen, dass Konvergenz bezüglich der $h_l$ hinreichend ist für vage Konvergenz, dann stimmen die vage Topologie und die durch die $h_l$ induzierte überein. Eine abzählbare Subbasis wäre dann $\{ev_{h_l}^{-1}((r,s)) \, | \, l \in \mathbb{N}, r,s \in \mathbb{Q}\}$.\\
Nehme also an, dass für eine Folge $(\mu_n) \subset M_+(E)$ für jedes $l \in \mathbb{N}$ eine Konstante $c_l \in [0,\infty)$ existiert, sodass $\mu_n(h_l) \to c_l$. Zeige, dass dann ein Maß $\mu \in M_+(E)$ existiert mit $\mu(h_l) = c_l \, \forall \, l \in \mathbb{N}$. Folgere dann $\mu_n \xrightarrow{v} \mu$. Wir zeigen zunächst, dass die Menge $\{\mu_n \, | \, n \in \mathbb{N} \}$ relativ kompakt ist und nutzen dafür die Charakterisierung $(ii)$ aus der vorherigen Proposition. Sei dafür $f \in C_K^+(E)$ beliebig und $F$ der kompakte Träger von $f$. Da $\mathcal{G}$ eine offene Überdeckung von $E$ bildet, gibt es also eine endliche Teilüberdeckung von $F$, aber da $\mathcal{G}$ per Annahme bereits abgeschlossen unter endlichen Vereinigungen ist, gilt $F \subset G_{i_0}$ für ein $i_0 \in \mathbb{N}$. Setze $0 \leq \beta < \infty$ als das Maximum von $f$. Dann gilt für jedes $k \in \mathbb{N}$:
\begin{align}
f \leq \beta \; 1_F \; \leq \beta \; 1_{G_{i_0}} \leq \beta \; 1_{\overline{G}_{i_0}} \leq \beta \; g_{i_0,k}
\end{align}
und damit
\begin{align}
\sup_{n \in \mathbb{N}} \mu_n(f) \leq \beta \; \sup_{n \in \mathbb{N}} \mu_n(g_{i_0,k}) < \infty,
\end{align}
da $\mu_n(h_l)$ für jedes $l$ konvergiert und somit beschränkt ist. Da also die Menge $\{\mu_n \, | \, n \in \mathbb{N} \}$ relativ kompakt ist, gibt es eine Teilfolge $\mu_{n_k}$, die vage gegen ein $\mu \in M_+(E)$ konvergiert. Daraus folgt, dass $\mu_{n_k}(h_l) \to \mu(h_l)$, aber andererseits gilt $\mu_{n_k}(h_l) \to c_l$ per Voraussetzung, also folgt $\mu(h_l) = c_l$. Das Maß $\mu$ wird durch die $c_l$ eindeutig bestimmt. Nutze nun, um $\mu_n \xrightarrow{v} \mu$ zu folgern, die folgende grundlegende Aussage aus der Topologie: Für eine Folge $(x_n)$ in einem topologischen Raum $X$ gilt $x_n \to x$ genau dann wenn für alle Teilfolgen $(x_{n_k})$ eine weitere Teilfolge $(x_{n_{k_j}})$ existiert mit $x_{n_{k_j}} \to x$. Damit folgt in unserem Fall aber sofort $\mu_n \xrightarrow{v} \mu$, da wir die obigen Argumente für eine beliebige Teilfolge wiederholen können und für diese eine konvergente Teilfolge erhalten, die jedes Mal gegen das gleiche, durch die $c_l$ eindeutig festgelegte $\mu$ konvergiert.\\


Eine Metrik $d$, die die Konvergenz bezüglich aller $h_l$ induziert und $(M_+(E),d)$ vollständig macht, ist
\begin{align}
d(\mu,\nu) := \sum_{l=1}^\infty 2^{-l} \left(1 - e^{- | \mu(h_l) - \nu(h_l)|} \right).
\end{align}
Zunächst einmal konvergiert die Reihe immer, da der Faktor in Klammern immer zwischen $0$ und $1$ liegt. Dass $d(\mu,\nu) \geq 0$, ist klar; und dass aus $d(\mu,\nu)=0$ folgt, dass $\mu = \nu$, ist nichts anderes als die Eindeutigkeitsaussage weiter oben. Dass $d$ symmetrisch ist, ist auch klar. Für die Dreiecksungleichung sind die wesentlichen Zutaten die gewöhnliche Dreiecksungleichung für den Betrag im Exponenten, und die für $x,y \geq 0$ gültige Ungleichung
\begin{align*}
1 - e^{-x} e^{-y} \leq 2 - e^{-x} -e^{-y},
\end{align*}
welche sich durch ''Wegstreichen'' der einen $1$, Multiplikation mit $e^x$ und Umsortieren für $y > 0$ auf die äquivalente Form $e^x \geq 1$ bringen lässt.\\
Beweis der Vollständigkeit: Sei $(\mu_n)$ Cauchy-Folge, also $d(\mu_n,\mu_m)\to 0$, d.\ h.\ für alle $l$ gilt $|\mu_n(h_l)-\mu_m(h_l)|\to 0$. Also sind die reellen Folgen $(\mu_n(h_l))_{n \in \mathbb{N}}$ Cauchy und konvergieren folglich gegen Konstanten $c_l \in [0,\infty)$, und nach dem vorangegangen Beweis konvergiert dann auch die Folge der Maße gegen ein $\mu \in M_+(E)$.
\end{proof}
%%%%%%%%%%%%%%%%%%%%%%%%%%%%%%%%%%%%%%%%%%%%%%%%%%%%%
\begin{proposition} (Vage Stetigkeit von Abbildungen zwischen Räumen von Maßen)\\
Seien $E$ und $E'$ zwei lokal kompakte, zweitabzählbare $T_2$-Räume und $T:E \to E'$ eine eigentliche Abbildung. Dann ist $\hat{T}:M_+(E) \to M_+(E'), \mu \mapsto \mu \circ T^{-1}$ stetig. Insbesondere ist also die Einschränkung von $\hat{T}$ auf $M_p(E)$, gegeben durch $\hat{T}\left( \sum_i \delta_{x_i} \right) = \sum_i \delta_{T(x_i)}$, stetig.
\end{proposition}

\begin{proof}
Seien $\mu_n,\mu \in M_+(E)$ mit $\mu_n \xrightarrow{v} \mu$ und $f \in C_K^+(E')$. Da $T$ eigentlich ist, gilt $f \circ T \in C_K^+(E)$. Also gilt 
\begin{align}
\hat{T}\left(\mu_n\right)(f) = \mu_n \circ T^{-1}(f) = \mu_n(f \circ T)  \to \mu(f \circ T) = \hat{T}\left(\mu\right)(f).
\end{align}
\end{proof}

\section*{Literatur}
\beginrefs
\bibentry{1}{\sc S.~Resnick}, 
``Extreme Values, Regular Variation and Point Processes''
{\it Springer-Verlag},
1987, pp.~139-150.
\bibentry{2}{\sc O.~Kallenberg}, 
``Random Measures''
{\it Elsevier Science \& Technology Books},
1983
\bibentry{3}{\sc H.~Dehling}, 
``Wahrscheinlichkeitstheorie I''
{\it Vorlesungsskript, Ruhr-Universität},
2018
\bibentry{4}{\sc B.~Basrak, H.~Planinić}, 
``A note on vague convergence of measures''
{\it arXiv preprint 1803.07024},
2018
\endrefs

\end{document}