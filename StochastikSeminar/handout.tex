\documentclass[twoside]{article}
\setlength{\oddsidemargin}{0. in}
\setlength{\evensidemargin}{-0 in}
\setlength{\topmargin}{-0. in}
\setlength{\textwidth}{7 in}
\setlength{\textheight}{8.4 in}
\setlength{\headsep}{0.75 in}
\setlength{\parindent}{0 in}
\setlength{\parskip}{0.05 in}

\usepackage{amsmath,amsfonts,graphicx,amsthm}
\usepackage[utf8]{inputenc}
\usepackage[ngerman]{babel}
\usepackage{enumerate}
\usepackage{xcolor}
\usepackage{todonotes}
\usepackage[left=2cm,right=2cm,top=3cm,bottom=2cm,]{geometry}

\newcounter{lecnum}
\renewcommand{\thepage}{\thelecnum-\arabic{page}}
\renewcommand{\thesection}{\thelecnum.\arabic{section}}
\renewcommand{\theequation}{\thelecnum.\arabic{equation}}
\renewcommand{\thefigure}{\thelecnum.\arabic{figure}}
\renewcommand{\thetable}{\thelecnum.\arabic{table}}

\setcounter{lecnum}{11}

\newcommand{\head}{
   \pagestyle{myheadings}
   \thispagestyle{plain}
   \newpage
   \setcounter{page}{1}
   \noindent
   \begin{center}
   \framebox{
      \vbox{\vspace{2mm}
    \hbox to 6.28in { {\bf Seminar zur Stochastik SoSe 2019
	\hfill Ruhr-Universität Bochum,  24.06.2019} }
       \vspace{4mm}
       \hbox to 6.28in { {\Large \hfill 11. Vage Konvergenz von Punktmaßen \hfill} }
       \vspace{2mm}
       \hbox to 6.28in { {\it Kurzzusammenfassung \hfill von Timo Schorlepp} }
      \vspace{2mm}}
   }
   \end{center}
}

\renewcommand{\cite}[1]{[#1]}
\def\beginrefs{\begin{list}%
        {[\arabic{equation}]}{\usecounter{equation}
         \setlength{\leftmargin}{2.0truecm}\setlength{\labelsep}{0.4truecm}%
         \setlength{\labelwidth}{1.6truecm}}}
\def\endrefs{\end{list}}
\def\bibentry#1{\item[\hbox{[#1]}]}

\newtheorem{theorem}{Satz}[lecnum]
\newtheorem{lemma}[theorem]{Lemma}
\newtheorem{proposition}[theorem]{Proposition}
\newtheorem{claim}[theorem]{Behauptung}
\newtheorem{corollary}[theorem]{Korollar}
\theoremstyle{definition}
\newtheorem{remark}[theorem]{Bemerkung}
\newtheorem{definition}[theorem]{Definition}
%\newenvironment{proof}{{\bf Beweis:}}{\hfill\rule{2mm}{2mm}}
%%%%%%%%%%%%%%%%%%%%%%%%%%%%%%%%%%%%%%%%%%%%%%%%%%%%%
\begin{document}

\head
%%%%%%%%%%%%%%%%%%%%%%%%%%%%%%%%%%%%%%%%%%%%%%%%%%%%%
\textbf{Einleitung:} In diesem Vortrag wollen wir in Vorbereitung auf die Beschäftigung mit schwacher Konvergenz von Punktprozessen den Raum aller Punktmaße auf $E$, bezeichnet mit $M_p(E)$, derart mit einem Konvergenzbegriff ausstatten und dementsprechend topologisieren (''vage Konvergenz/Topologie''), dass $M_p(E)$ zu einem vollständigen, separablen metrischen Raum wird. Dies ist genau das benötigte Setting für die Diskussion von schwacher Konvergenz von Maßen auf $M_p(E)$ (vgl.\ Vorlesung Wahrscheinlichkeitstheorie 1, bspw.\ Skript [3]). Wir werden dabei alle Resultate für den Raum \textit{aller} Radon-Maße $M_+(E)$ formulieren und entsprechende Aussagen für die, wie sich zeigen wird, abgeschlossene Teilmenge $M_p(E)$ folgern. Nach einer Einführung der vagen Topologie werden wir zunächst ein Portmanteau-Theorem zur Charakterisierung von vager Konvergenz, analog zu ähnlichen Resultaten für die schwache Konvergenz, beweisen, und am Ende eine abzählbare Basis der vagen Topologie und eine zugehörige Metrik auf $M_+(E)$ zu finden. Die Darstellung folgt größtenteils [1].
%%%%%%%%%%%%%%%%%%%%%%%%%%%%%%%%%%%%%%%%%%%%%%%%%%%%%
\begin{definition}
(Setting)\\
Sei $E$ ein lokal kompakter, zweitabzählbarer $T_2$-Raum mit Borel-$\sigma$-Algebra $\mathcal{E}$. Bezeichne mit $M_+(E)$ die Menge aller Radon-Maße auf E und wähle
\begin{align}
\mathcal{M}_+(E):=\sigma\left(\left\{\left\{\mu \in M_+(E)|\mu(f) \in B \right\}|f \in C_K^+(E),B \in \mathcal{B}\left[0,\infty \right) \right\}\right)
\end{align}
als $\sigma$-Algebra auf $M_+(E)$, wobei $C_K^+(E):=\left\{ f:E \to [0,\infty) | f \text{ stetig, } \text{supp}(f) \text{ kompakt}\right\}$ die Menge aller stetigen, nichtnegativen Funktionen mit kompaktem Träger auf $E$ ist und $ev_f(\mu) = \mu(f) := \int_E f \; \mathrm{d}\mu$.
\end{definition}

\begin{lemma} (Existenz von Höckerfunktionenfolgen, vgl.\ Lemma 10.2 in [3])\\
(a) Sei $K \subseteq E$ kompakt. Dann existieren kompakte Mengen $K_n \downarrow K$ und eine monoton fallende Folge $(f_n)_{n \in \mathbb{N}}$ mit $f_n \in C_K^+(E) \, \forall \, n \in \mathbb{N}$ mit $1_K \leq f_n \leq 1_{K_n} \downarrow 1_K$.\\
(b) Sei $G \subseteq E$ offen und relativ kompakt. Dann existieren offene und relativ kompakte Mengen $G_n \uparrow G$ und eine monoton wachsende Folge $(g_n)_{n \in \mathbb{N}}$ mit $g_n \in C_K^+(E) \, \forall \, n \in \mathbb{N}$ mit $1_G \geq g_n \geq 1_{G_n} \uparrow 1_G$.
\end{lemma}

\begin{proposition} (vgl.\ Satz 10.3 in [3])\\
$C_K^+(E)$ ist eine trennende Familie für $M_+(E)$, d.\ h.\ für $\mu, \nu \in M_+(E)$ gilt: 
\begin{align}
\forall \, f \in C_K^+(E) : \, \mu(f) = \nu(f) \Longrightarrow \mu = \nu.
\end{align}
\end{proposition}

\begin{definition} 
(Vage Konvergenz)\\
Sei $(\mu_n)_{n \in \mathbb{N}}$ eine Folge in $M_+(E)$ und $\mu \in M_+(E)$. Wir sagen, dass die Folge $(\mu_n)$ vage gegen $\mu$ konvergiert (geschrieben $\mu_n \xrightarrow{v} \mu$), wenn
$
\forall f \in C_K^+(E): \; \mu_n(f) \xrightarrow{} \mu(f)
$.\\
Wir topologisieren $M_+(E)$ unter diesem Konvergenzbegriff, das heißt eine Subbasis ist gegeben durch Mengen der Form $\left\{\mu \in M_+(E) | s < \mu(f) < t \right\}$ für $f \in C_K^+(E)$ und $s,t \in \mathbb{R}, s < t$, und die vage Topologie ist damit die durch alle Auswertungsabbildungen $ev_f:M_+(E) \to [0,\infty), \mu \mapsto \mu(f)$ induzierte Topologie.
\end{definition}

\begin{proposition}
Es gilt $\mathcal{M}_+(E) = \mathcal{B}(M_+(E))$.
\end{proposition}

\begin{theorem} (Portmanteau-Theorem für vage Konvergenz, vgl.\ Satz 10.6 in [3] für schwache Konvergenz)\\
Sei $(\mu_n)_{n \in \mathbb{N}}$ eine Folge in $M_+(E)$ und $\mu \in M_+(E)$. Dann sind folgende Aussagen äquivalent:
\begin{enumerate}[(i)]
\item $\mu_n \xrightarrow{v} \mu$
\item $\mu_n(B) \to \mu(B)$ für alle relativ kompakten Mengen $B \in \mathcal{E}$ mit $\mu(\partial B) = 0$.
\item $\limsup_{n \to \infty} \mu_n(K) \leq \mu(K)$ für alle kompakten $K \in \mathcal{E}$ und $\liminf_{n \to \infty} \mu_n(G) \geq \mu(G)$ für alle offenen, relativ kompakten $G \in \mathcal{E}$.
\end{enumerate}
\end{theorem}
\newpage
\begin{proposition} (Vage Konvergenz von Punktmaßen)\\
Sei $(m_n)_{n \in \mathbb{N}}$ eine Folge in $M_p(E)$ und $m \in M_p(E)$. Dann sind folgende Aussagen äquivalent:
\begin{enumerate}[(i)]
\item $m_n \xrightarrow{v} m$.
\item Für alle $B \in \mathcal{E}$ relativ kompakt mit $m(\partial B) = 0$ gibt es für $n \geq n_0(B)$ eine Nummerierung der Punkte von $m_n$ und $m$ in $B$ mit
\begin{align}
m_n(\cdot \cap B) = \sum_{i=1}^p \delta_{x_i^{(n)}} \; ; \quad m(\cdot \cap B) = \sum_{i=1}^p \delta_{x_i} \label{eq:pktmass}
\end{align}
und $(x_1^{(n)}, \cdots , x_p^{(n)}) \to (x_1, \cdots , x_p)$ in $E^p$.
\end{enumerate}
\end{proposition}

\begin{lemma} (Approximation kompakter Mengen durch randlose Mengen)\\
Sei $K \subset E$ kompakt und $\mu \in M_+(E)$. Dann exisitert eine reelle Folge $\epsilon_n \downarrow 0$ mit $K^{\epsilon_n} \downarrow K$ und $\mu(\partial K^{\epsilon_n} ) = 0 \, \forall n$.
\end{lemma}

\begin{proposition} 
Die Menge der Punktmaße $M_p(E)$ ist vage abgeschlossen in  $M_+(E)$.
\end{proposition}

\begin{proposition}(Relative Kompaktheitskriterien)\\
Sei $M \subseteq M_+(E)$ oder $M \subseteq M_p(E)$. Dann sind äquivalent:
\begin{enumerate}[(i)]
\item $M$ ist vage relativ kompakt.
\item $\sup_{\mu \in M} \mu(f) < \infty$  für alle $\forall f \in C_K^+(E)$.
\item $\sup_{\mu \in M} \mu(B) < \infty$ für alle relativ kompakten Mengen $B \in \mathcal{E}$.
\end{enumerate}
\end{proposition}

\begin{theorem} (Metrisierbarkeit)\\
$M_+(E)$ und $M_p(E)$, ausgestattet mit der vagen Topologie, sind metrisierbar zu vollständigen, separablen metrischen Räumen.
\end{theorem}

\begin{proposition} (Vage Stetigkeit von Abbildungen zwischen Räumen von Maßen)\\
Seien $E$ und $E'$ zwei lokal kompakte, zweitabzählbare $T_2$-Räume und $T:E \to E'$ eine eigentliche Abbildung. Dann ist $\hat{T}:M_+(E) \to M_+(E'), \mu \mapsto \mu \circ T^{-1}$ stetig. Insbesondere ist also die Einschränkung von $\hat{T}$ auf $M_p(E)$, gegeben durch $\hat{T}\left( \sum_i \delta_{x_i} \right) = \sum_i \delta_{T(x_i)}$, stetig.
\end{proposition}

\section*{Literatur}
\beginrefs
\bibentry{1}{\sc S.~Resnick}, 
``Extreme Values, Regular Variation and Point Processes''
{\it Springer-Verlag},
1987, pp.~139-150.
\bibentry{2}{\sc O.~Kallenberg}, 
``Random Measures''
{\it Elsevier Science \& Technology Books},
1983
\bibentry{3}{\sc H.~Dehling}, 
``Wahrscheinlichkeitstheorie I''
{\it Vorlesungsskript, Ruhr-Universität},
2018
\bibentry{4}{\sc B.~Basrak, H.~Planinić}, 
``A note on vague convergence of measures''
{\it arXiv preprint 1803.07024},
2018
\bibentry{5}
Ausführlichere Notizen zum Vortrag:\\https://github.com/TimoSchorlepp/MiscCoursework/tree/master/StochastikSeminar
\endrefs

\end{document}