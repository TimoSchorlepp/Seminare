\documentclass[twoside]{article}
\setlength{\oddsidemargin}{0. in}
\setlength{\evensidemargin}{-0 in}
\setlength{\topmargin}{-0. in}
\setlength{\textwidth}{7 in}
\setlength{\textheight}{8.4 in}
\setlength{\headsep}{0.75 in}
\setlength{\parindent}{0 in}
\setlength{\parskip}{0.05 in}

\usepackage{amsmath,amsfonts,graphicx,amsthm}
\usepackage[utf8]{inputenc}
\usepackage[ngerman]{babel}
\usepackage{enumerate}
\usepackage{xcolor}
\usepackage{todonotes}
\usepackage{csquotes}
\usepackage{physics}
\usepackage{subcaption}
\usepackage{booktabs}
\usepackage[left=2cm,right=2cm,top=3cm,bottom=2cm,]{geometry}
\graphicspath{{figures/}}
\newcommand\independent{\protect\mathpalette{\protect\independenT}{\perp}}
\def\independenT#1#2{\mathrel{\rlap{$#1#2$}\mkern2mu{#1#2}}}

\newcounter{lecnum}
\renewcommand{\thepage}{\thelecnum-\arabic{page}}
\renewcommand{\thesection}{\thelecnum.\arabic{section}}
\renewcommand{\theequation}{\thelecnum.\arabic{equation}}
\renewcommand{\thefigure}{\thelecnum.\arabic{figure}}
\renewcommand{\thetable}{\thelecnum.\arabic{table}}

\setcounter{lecnum}{11}

\newcommand{\head}{
   \pagestyle{myheadings}
   \thispagestyle{plain}
   \newpage
   \setcounter{page}{1}
   \noindent
   \begin{center}
   \framebox{
      \vbox{\vspace{2mm}
    \hbox to 6.28in { {\bf Seminar zur Numerik WiSe 19/20
	\hfill Ruhr-Universität Bochum,  16.01.2020} }
       \vspace{4mm}
       \hbox to 6.28in { {\Large \hfill 11. Konvergenzsätze für Markov-Ketten; MCMC-Methoden \hfill} }
       \vspace{2mm}
       \hbox to 6.28in { {\it Handout \hfill von Timo Schorlepp} }
      \vspace{2mm}}
   }
   \end{center}
}

\renewcommand{\cite}[1]{[#1]}
\def\beginrefs{\begin{list}%
        {[\arabic{equation}]}{\usecounter{equation}
         \setlength{\leftmargin}{2.0truecm}\setlength{\labelsep}{0.4truecm}%
         \setlength{\labelwidth}{1.6truecm}}}
\def\endrefs{\end{list}}
\def\bibentry#1{\item[\hbox{[#1]}]}

\newtheorem{theorem}{Satz}[lecnum]
\newtheorem{lemma}[theorem]{Lemma}
\newtheorem{proposition}[theorem]{Proposition}
\newtheorem{claim}[theorem]{Behauptung}
\newtheorem{corollary}[theorem]{Korollar}
\theoremstyle{definition}
\newtheorem{remark}[theorem]{Bemerkung}
\newtheorem{definition}[theorem]{Definition}
\newtheorem{example}[theorem]{Beispiel}
%\newenvironment{proof}{{\bf Beweis:}}{\hfill\rule{2mm}{2mm}}
%%%%%%%%%%%%%%%%%%%%%%%%%%%%%%%%%%%%%%%%%%%%%%%%%%%%%
\begin{document}

\head
%%%%%%%%%%%%%%%%%%%%%%%%%%%%%%%%%%%%%%%%%%%%%%%%%%%%%
\textbf{Erinnerung an wichtige Definitionen und Sätze:}\\
$(\Omega, {\cal F}, P)$ sei W-Raum, $Z$ bezeichne eine endliche Menge mit der Potenzmenge als $\sigma$-Algebra, $(X_t)_{t \in \mathbb{N}_0}$ sei zeit-diskreter stochastischer Prozess mit $X_t:\Omega \to Z$ messbar $\forall \, t$, und $({\cal F}_t)_{t \in \mathbb{N}_0}$ sei die zugehörige Filtration auf $\Omega$, d.h.\ ${\cal F}_t = \sigma(X_0 , \dots , X_t) \subset {\cal F}$. Eine Abbildung $T: \Omega \to \mathbb{N}_0 \cup \{+\infty\} =:\mathbb{N}_{\infty}$ heißt Stoppzeit bzgl.\ $(X_t)$, falls $\{T=t\} \in {\cal F}_t \forall \, t \in \mathbb{N}_{\infty}$ (mit ${\cal F}_{\infty} = \sigma \left(X_0,X_1,\dots\right)$).\\

$(X_t)$ heißt homogene Markov-Kette, falls $P(X_{n+1}=z_{n+1}|(X_0, \dots , X_n) = (z_0, \dots, z_n)) = P(X_{n+1}=z_{n+1}|X_n =z_n)$ für alle $n \in \mathbb{N}$ und $z_0, \dots , z_{n+1} \in Z$ mit $P((X_0,\dots,X_n)=(z_0, \dots,z_n))>0$, sowie $Q_{z,z'} := P(X_{n+1}=z'|X_n=z) \neq Q_{z,z'}(n)$ für $P(X_n = z) > 0$. Existiert $z_0 \in Z$ mit  $P(X_n = z_0) = 0$ für alle $n$, so setze $Q_{z_0,z'} = \delta_{z_0,z'}$. Dann gilt für die so konstruierte Matrix $Q \in \mathbb{R}^{Z \times Z}$: $Q_{z,z'} \geq 0$ für alle $z,z' \in Z$, und für die Zeilensummen gilt $\sum_ {z' \in Z} Q_{z,z'} = 1$, womit $Q$ per Definition eine stochastische Matrix ist. Ist $T$ fast sicher endlich, so ist $(X_{T+n})$ ebenfalls eine Markov-Kette mit Übergangsmatrix $Q$. Für die Verteilungen $\mu^{(n)}$ von $X_n$ gilt $\mu^{(n+l)}=\mu^{(n)} \cdot Q^l$. $Q$ heißt irreduzibel, falls $\forall z,z' \in Z : \exists n>0:(Q^n)_{z,z'}>0$. $Q$ heißt aperiodisch, falls $\forall z \in Z:  \text{ggT}(\{n>0|(Q^n)_{z,z}>0\})=1$. Ist $Q$ irreduzibel und $Q_{z,z}>0$ für ein $z \in Z$, so ist $Q$ aperiodisch. Ferner ist $Q$ genau dann irreduzibel und aperiodisch, wenn $Q^n$ positiv ist für ein $n>0$, d.h.\ $(Q^n)_{z,z'}>0 \forall z,z' \in Z$. Für die Stoppzeit $T_z = \inf \{n >0| X_n=z \}$ für $z \in Z$ (\textit{Eintrittszeit}) gilt, falls $Q$ irreduzibel und aperiodisch ist, $ET_z<\infty$ und insbesondere ist $T_z$ fast sicher endlich.\\

Ein Wahrscheinlichkeitsvektor $\mu \in \mathbb{R}^Z$ (d.h.\ $\mu_z \geq 0 \forall z \in Z$ und $\sum_{z \in Z} \mu_z=1$) heißt stationäre Verteilung von $Q$, falls $\mu = \mu \cdot Q$. Ist $Q$ irreduzibel mit einer stationären Verteilung $\mu$, so ist $\tilde{Q}:=(I+Q)/2$ (\textit{lazy version} von $Q$) irreduzibel und aperiodisch mit stationärer Verteilung $\mu$. Da $\mu$ stationäre Verteilung von $Q$ ist genau dann wenn $\mu_z = \sum_{z' \in Z} \mu_{z'} Q_{z',z}$, folgt sofort, dass, falls $\mu$ die \textit{detailed-balance}-Gleichung $\mu_z Q_{z,z'} = \mu_{z'} Q_{z',z}$ $\forall z,z' \in Z$ erfüllt, $\mu$ auch stationäre Verteilung von $Q$ ist. Abschließend wurde gezeigt, dass für irreduzible und aperiodische $Q$ immer eine positive stationäre Verteilung existiert.\\

%%%%%%%%%%%%%%%%%%%%%%%%%%%%%%%%%%%%%%%%%%%%%%%%%%%%%
\textbf{Das hard core model:}\\
Wir betrachten einen ungerichteten Graphen $(E,K)$ mit $\abs{E} < \infty$, $K \neq \emptyset$ und $Z \subset \tilde{Z}=\{0,1\}^E$, sodass für $z \in Z$ gilt, dass $z(e)=z(e')=1 \Rightarrow \{e,e'\} \notin K$. Die gesuchte stationäre Verteilung sei die Gleichverteilung auf $Z$, und eine natürliche Frage ist zum Beispiel: Wie groß ist die mittlere Besetzungszahl, d.h.\ was ist $E_\mu(f)$ für $f:Z \to \mathbb{R}, z \mapsto 1/\abs{E} \sum_ {e \in E} z(e)$? Das Problem mit einer direkten Simulation basierend auf der Verwerfungsmethode ist, dass die Akzeptanzwahrscheinlichkeit im Allgemeinen exponentiell klein in $\abs{E}$ ist.\\

%%%%%%%%%%%%%%%%%%%%%%%%%%%%%%%%%%%%%%%%%%%%%%%%%%%%%
\textbf{Das Ising-Modell:}\\
Es sei wieder $(E,K)$ ein ungerichteter Graph mit $\abs{E} < \infty$, $K \neq \emptyset$ und $Z =\{-1,1\}^E$. Das Ziel ist es, Erwartungswerte bezüglich einer Boltzmann-Verteilung $\mu^\beta$ auf $Z$ mit $\beta>0$ zu berechnen, d.h.\ $\mu^\beta_z = 1/C_\beta \exp(-\beta H(z))$ für $z \in Z$ mit der Zustandssumme $C_\beta$ als Normierungsfaktor und der Hamilton-Funktion $H:Z\to \mathbb{R}$ von der allgemeinen Form $H(z) = \sum_{e \in E} h_e(z(e)) + \sum_{\{e,e'\}\in K} h_{\{e,e'\}}(z(e),z(e'))$, wobei die Funktionen $h_e:\{-1,1\} \to \mathbb{R}$ bspw.\ ein äußeres, ortsabhängiges Magnetfeld modellieren, und die Funktionen $h_{\{e,e'\}}:\{-1,1\}^2\to \mathbb{R}$ Wechselwirkungen benachbarter Spins modellieren. Ein interessante Frage ist beispielsweise, wie der Erwartungswert der absoluten Magnetisierung $f:Z \to \mathbb{R}, z \mapsto 1/\abs{E} \abs{\sum_ {e \in E} z(e)}$ im Ising-Modell ohne äußeres Magnetfeld von der (inversen) Temperatur $\beta$ abhängt. Das Problem mit naiven Monte-Carlo-Simulationen wie
\begin{align*}
\frac{\sum_{i=1}^n f(X_i) e^{-\beta H(X_i)}}{\sum_{i=1}^n e^{-\beta H(X_i)}}
\end{align*}
für gleichverteilte $Z$-wertige uiv.\ ZV $X_i$ ist die große Varianz für große $\beta$, wenn die Boltzmann-Verteilung von wenigen Zuständen nahe an den Minima von $H$ dominiert ist. 
%%%%%%%%%%%%%%%%%%%%%%%%%%%%%%%%%%%%%%%%%%%%%%%%%%%%%
\newpage
\textbf{In diesem Vortrag:}
Theoretische Grundlagen für die Markov-Chain-Monte-Carlo-Methode (MCMC) + Anwendung auf obige Beispiele.
\begin{theorem}(Konvergenz gegen stationäre Verteilungen; Satz 6.28 in [1])\\
Sei $Q \in \mathbb{R}^{Z \times Z}$ eine irreduzible und aperiodische stochastische Matrix und $\mu \in \mathbb{R}^Z$ eine stationäre Verteilung von $Q$. Dann existieren (feste) Konstanten $c>0$ und $\alpha \in (0,1)$ mit
\begin{align}
\max_{z \in Z} \abs{(\mu^{(0)} \cdot Q^n)_z - \mu_z} \leq c \cdot \alpha^n
\end{align}
für alle Wahrscheinlichkeitsvektoren $\mu^{(0)} \in \mathbb{R}^Z$ und alle $n \in \mathbb{N}$.
\end{theorem}
\begin{theorem}(Ergodensatz für homogene Markov-Ketten; Satz 6.30 in [1])\\
Sei $(X_n)_{n \in \mathbb{N}_0}$ eine homogene Markov-Kette mit endlichem Zustandsraum $Z$,  irreduzibler und aperiodischer Übergangsmatrix $Q$ und stationärer Verteilung $\mu \in \mathbb{R}^Z$. Dann gilt für jede Abbildung $f:Z\to \mathbb{R}$
\begin{align}
\frac{1}{n} \sum_{i=1}^n f(X_i) \xrightarrow[n  \to \infty]{\text{f.s.}} E_\mu(f).
\end{align}
\end{theorem}

\textbf{Verbleibende Aufgabe:} Wie konstruiert man nun im Allgemeinen eine irreduzible und aperiodische stochastische Matrix zu einer gegebenen Verteilung $\mu \in \mathbb{R}^Z$, sodass die zugehörige Markov-Kette mit möglichst wenig Aufwand zu simulieren ist? Starte dazu mit einer \textit{beliebigen} irreduziblen und symmetrischen stochastischen Matrix $\tilde{Q} \in \mathbb{R}^{Z \times Z}$. Wähle dann für $z \neq z'$ sogenannte Akzeptanzwahrscheinlichkeiten $\alpha_{z,z'} \in (0,1]$, die bezüglich der stationären Verteilung $\mu$ die detailed-balance-artigen Gleichungen
\begin{align}
\mu_z \cdot \alpha_{z,z'} = \mu_{z'} \cdot \alpha_{z',z}
\end{align}
erfüllen. Dann ist die Matrix $Q$, definiert durch $Q_{z,z'}=\tilde{Q}_{z,z'} \cdot \alpha_{z,z'}$ für $z \neq z'$ und $Q_{z,z} = 1 - \sum_{z' \neq z} Q_{z,z'}$, ebenfalls eine stochastische irreduzible Matrix mit stationärer Verteilung $\mu$. Ist $\tilde{Q}$ aperiodisch, so auch $Q$.\\

Zwei spezielle Wahlen von Akzeptanzwahrscheinlichkeiten, die für beliebige $\mu$ die detailed-balance-Gleichung erfüllen, sind der Metropolis-Algorithmus [5] mit
\begin{align}
\alpha_{z,z'} = \min\left(1,\frac{\mu_{z'}}{\mu_z}\right)
\end{align}
und der Gibbs-Sampler
\begin{align}
\alpha_{z,z'} = \frac{\mu_{z'}}{\mu_z+\mu_{z'}}.
\end{align}
%%%%%%%%%%%%%%%%%%%%%%%%%%%%%%%%%%%%%%%%%%%%%%%%%%%%%
\section*{Literatur}
\beginrefs
\bibentry{1}{\sc T.~Müller-Gronbach, E.~Novak, K.~Ritter}, 
``Monte Carlo-Algorithmen''
{\it Springer-Verlag},
2012.
\bibentry{2}{\sc F.~Schwabl}, 
``Statistische Mechanik''
{\it Springer-Verlag},
2006.
\bibentry{3}{\sc M.~E.~J.~Newman, G.~T.~Barkema}, 
``Monte Carlo Methods in Statistical Physics''
{\it Oxford University Press},
2002.
\bibentry{4}{\sc D.~A.~Levin, Y.~Peres}, 
``Markov Chains and Mixing Times''
{\it American Mathematical Soc.},
2017.
\bibentry{5}{\sc N.~Metropolis, A.~W.~Rosenbluth, M.~N.~Rosenbluth, A.~H.~Teller, E.~Teller}, 
``Equation of State Calculations by Fast Computing Machines''
{\it  J. Chem. Phys. 21, 1087},
1953.
\bibentry{6}{\sc L.~Onsager}, 
``Crystal Statistics. I. A Two-Dimensional Model with an Order-Disorder Transition''
{\it Phys. Rev. 65, 117},
1944.
\bibentry{7}{} Ausführlichere Notizen und der im Vortrag gezeigte Code für das hard core model und das Ising-Modell finden sich unter https://github.com/TimoSchorlepp/MiscCoursework/tree/master/MCSeminar
\endrefs
\end{document}